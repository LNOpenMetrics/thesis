\chapter{Bitcoin}

\section{Introduction}

Bitcoin is a digital currency that uses \emph{blockchain} technology to track
and verify transactions. It was created in 2009 by an unknown individual or
group using the pseudonym \emph{Satoshi Nakamoto}.
Bitcoin is decentralized, meaning that it is not controlled by any government
or institution. Instead, it relies on a network of computers around the world
to process and verify transactions.
The technology used is known with the name blockchain, and it is a transparent
and secure record of all transactions that have
ever occurred on the network.

\section{Bitcoin Basics}
\label{sec:basics}

One of the key features of Bitcoin is its limited supply. There will only ever
be 21 million bitcoins in existence, and the rate at which new bitcoins are
created is gradually decreasing. This built-in scarcity is one of the factors
that makes Bitcoin valuable.

However, the innovative idea proposed by Bitcoin was not a virtual currency that can be
exchange on internet, but Bitcoin solves the problem of authority inside this kind
of system, through a proof of work protocol that was first proposed in a paper
by Adam Back in 1997 called \quotes{Hashcash - A Denial of Service Counter-Measure}.

In the context of Bitcoin, proof of work is a system that is used to secure the
blockchain and prevent fraud. It is a mathematical algorithm that requires computers
to perform a certain amount of computational work in order to create a new
block of transactions on the blockchain.

Therefore, the core ideas that make Bitcoin a solid cryptocurrency will limit the
usage of Bitcoin, and a well known limitation of the Bitcoin network is the number of
transactions that it can process per second. The current maximum is around 7 transactions
per second, which is relatively low compared to other payment systems.

This has led to congestion on the network such as in April of 2022 where the world start to
see the power of some cryptocurrency like bitcoin, and some of the limitation of it in the usage.

In fact, in the April of 2022 the number of transaction increase in the Network, and this
impact of time confirmation time of the blockchain and also on the fee cost that the miner
take to take in consideration a particular transaction.

The reason for this limitation is that the size of each block on the Bitcoin blockchain is
limited to 1 megabyte. This means that there is a physical limit to the number of transactions
that can be included in a single block. In order to increase the number of transactions
that the network can handle, the size of each block would need to be increased. However,
this is a contentious issue and there is currently no consensus on how to do this.

Overall, the limited number of transactions and the scalability issues on the Bitcoin network
are significant challenges that need to be addressed in order for the network to continue
to grow and evolve. There are various proposals and solutions being discussed,
but it remains to be seen how these issues will be resolved. In the Section \ref{sec:lightning_network}
is described a solution proposed to this problem.


\section{Bitcoin Script}

- How the transactions is a immutable concept and how the change transaction happens
- How a transaction is verified and spendable by the correct person/people

- Introduce an example that can be evolved long all the paper.

At the core of Bitcoin Protocol there is a concept of transaction, and it is used to transfer value
between users on the network. When a user is identified by a private and public wants, and the own of a
Bitcoin transaction is expressed in the he possibility to unlock a Unspendable transaction output.

The Bitcoin protocol do not express any concept of Bitcoin account, but there is a concept of address
derived from a \emph{Bitcoin script} program encoded inside the transaction.

A Bitcoin script is a simple, stack-based programming language used to encode the rules for a specific
transaction or set of transactions on the Bitcoin network. Scripts are used to determine how the
funds in a particular transaction can be spent, as well as to enforce certain rules on the network such as the owners
of the bitcoin transactions.

\subsection{Bitcoin Script Basics}

- Bitcoin Script basics, introduce how the bitcoin scrips works, and how may
kind of scripts exists
- Introduce the Bitcoin no standard
- Make a small zoom of what happens with the Bitcoin Script inside the example

\subsection{HTLC: Hash Time Locked Contract}
\label{sec:htlc_intro}

- Explain how the Transaction with a not standard script are used in lightning;
- Explain how the Transaction are replaced and managed in lightning at bitcoin level;
- Tell about bitcoin script and the segregated witness update;
- Use the example before but with lightning, and highlight the benefit and the downside.
