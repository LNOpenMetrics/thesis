\chapter{Conclusion}
\label{chap:conclusion}

The work done during this research period has provided the opportunity
to analyze the Lightning Network community and development process,
with the aim of developing a framework that makes Lightning Network
research more accessible to individuals who are not directly
involved in protocol development.

Furthermore, this work has highlighted the existence of other
frameworks such as lnmetrics, which aims to collect real data from
real users and provide this data directly to researchers. This will
drastically improve the way in which research on the Lightning Network is
conducted, and will enable bachelor's theses to explore the {\LN}
with real data, as well as propose variations of the current proposed metrics.

In terms of future development, there are several priorities.
One of the high-priority tasks is the development of a plugin system
in the lnmetrics analysis server, with a design that allows researchers to easily
integrate new metrics within the lnmetrics analysis server architecture,
which manages access to the database and resolves GraphQL queries for the new metrics.

Additionally, we hope that this solution will be adopted by universities with {\LN}
research groups, and that they will contribute new metrics to the LN Metrics RFC,
or provide future evolutions to our case study, such as conducting a deep analysis
of channels and creating a node scoring system based on the scores of our channels,
as well as having a channel score that will help with channel management tools.

Finally, we hope to receive contributions from Lightning Network users that can
help us to improve our dataset, which we aim to share with all those interested.
