\chapter{Conclusion}

The work done during this reseach period give the possibility
to analyze the {\LN} community and development process in order to allow
to develop a framework to make more accessible the reseach on the Lightning
Network done by people that are node directly involved to the  {\LN} protocol
development.

In addition, this work give some hints while analyzing the state of the art
that there are other framework like lnmetrics that aims to collect real data
from real user and provide this data directly to the researcher.
This will drasticly improve the way that a research will try to choose a topic
for a Lightning Network reseach in the near future, and also give the possibility
to bhecheloer thesis to explore the lightning network with real data, and propose
variation of the current proposed metrics.

In terms of future development of this thesis there are planetary chooise, but
a few work we consider with hight priority such as the development of a plugin system
in the lnmetrics analsys server with a design that allow the reseach to write in
an easy way to integration of a new metric inside the lnmetrics analsysis server
architecture that manage the access to the database and resolve GraphQL query
for the a new metrics.

In addition, we hope that this solution will be adopted by University that will
have Lightning Network research group and receive contribution with new metrics
in the LN Metrics RFC or with some future evolution to our case study such as
make a deep analysis of the channel that a node and create a node scoring based on the
score of our channels as well as have a channel score that will help the channel
management for the user.

Finally we hope to receive contribution with data from a {\LN} users that
help to improve our dataset to share with all the people interested.

In conclusion, from the complexity and the velocity of the {\LN} network protocol
development there is a need of more space for imrpovment to our architecture and how
to support a new lightning node implementation as well receive some protocol improvment
that help us to answer to some of the question
that we leeve open during the analsys such as the mitication of the channel jamming with a
local score that do not penalize good {\LN} nodes.
