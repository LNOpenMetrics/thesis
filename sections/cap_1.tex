\setcounter{page}{1}

\chapter{Lightning Network}

\newcommand{\noteOnBitcoinNaming}[0]{
    \footnote{
        From now till the end of the document we will use the word Bitcoin Protocol and
        Bitcoin with capitalized for identify the protocol, and the word bitcoin not capitalized to identify the currency
    }
}

\newcommand{\noteOnCLNImpl}[0]{
    \footnote{
    core lightning is one of the main implementation today, and it is available at
    \href{https://github.com/ElementsProject/lightning}{https://github.com/ElementsProject/lightning}
    }
}



\section{Introduction}

The \emph{Lightning Network} is a peer-to-peer protocol that propose a scalable solution for Bitcoin\noteOnBitcoinNaming where one
of the goal of the proposal is to be able to execute micro transaction with it.
In fact, Bitcoin does not scale for a sequence of reason, and one of them is the number of
maximum rate transaction that the network can process.\\
Therefore, with the awareness that some of these limitation that are feature in Bitcoin, the lightning network try to solve
the scalability problem by offer an \emph{off-chain} solution that enable peers to exchange partial
signed transaction without violate the Bitcoin rules.\\
Lightning Network is proposed for the first time in the 2015 in a draft paper \cite{lightning-network-paper},
and then implemented in the 2018 with the \emph{c-lightning}\noteOnCLNImpl (aka \emph{core lightning}) implementation.
However, the protocol is evolving over the years by a process of standardization\cite{lightning-bolts}, and the paper is considered
only a written idea on how to build an off-chain solution without any definition of the current protocol state machine.

\section{Off-Chain Protocol}

The Bitcoin network has scalability problem in confront of the modern payment system that the people use daily.
In particular, these problems are related to the space required by the blockchain to store each transaction,
and the relative long time that is required to consider a transaction confirmed in the blockchain.

In fact, the payment network Visa achieved around 65,000\cite{visa-sheet} transactions per second, and currently,
Bitcoin supports around 7 transactions per second with a required space estimated of 1 megabyte per block. Considering,
an average of 300 bytes per bitcoin transaction and assumed unlimited block sizes, the space for store the same number
of transactions is estimated in the order of terabytes\cite{lightning-network-paper}.

Therefore, in order to use the Bitcoin protocol as an alternative of the current system for a subset of
problem, such as the accessibility to the modern world to people that live in not developed places (e.g Africa),
there is the need to scale the protocol.

This kind of problem is almost considered a well studied problem, and different solution are given in different
way, that most of the time required a hard fork (See Section \ref{sec:hard_vs_soft} for the definition of hard fork) of the protocol itself.
One of the most popular hard fork example is given by the Bitcoin Cash that increase the block size to be able to insert
inside a block more transaction.
However, in the Bitcoin ecosystem (based on the Bitcoin Core implementation) the idea of a hard fork is considered an
end of choice alternative, and some of the idea to solve the scalability of the protocol are proposed through Bitcoin Script trick
that sometimes required a soft fork in the base protocol itsef.

\section{Lightning Network Basics}

The actual Lightning Network is more complicated from what was described in \cite{lightning-network-paper}, and it is
evolved a lot from the first time that the paper came out. However, the idea described in the paper was an evolution
of some previous ideas regarding the possibility to have payment channels with Bitcoin.

\subsection{Payment Channel}

A Payment Channel also know as Micropayment Channel is class of techniques designed to allow users to make multiple
Bitcoin transactions without commiting all of the transactions to the Bitcoin blockchain. In a typical payment channel,
only two transactions are added to the blockchain but an unlimited or nearly unlimted number of payments
can be made between the participants.

In fact, the concept of replace an old transaction with the new one is proposed from \emph{Satoshi Nakamoto}\footnote{TODO: described who this guys is} in the mainlist\cite{payment-channels-satoshi}, and
the basic concept to implement this feature was present from the version 0.0.1 of Bitcoin core. However, this solution
was not completely safe because with this method there is the possibility to steal funds from the other party or parties.\\

Later, there is a new proposal to implement this kind of concept called \emph{Spillman-style payment channels} is proposed, and it is
implemented inside a \emph{bitcoinj}\cite{bitcoinj-impl}. In this proposal, Spillman payment channels are unidirectional where
there is a payer and a payee, and it is not possible to transfer money back in the reverse direction, and the payment
channels expire after a specific time, and the receiver needs to close the channel before the expiration. However, this
proposal has malleability problem solved later by the \emph{BIP65}\cite{bip65} proposal that include a new Bitcoin Script OP code
called \emph{OP\_CHECKLOCKTIMEVERIFY} that unlock a new version of the \emph{Spillman-style payment channels}.

While the idea of the payment channels start to be interesting around the Bitcoin community, two new proposal came out that are
\emph{Poon-Dryja payment channels}\cite{lightning-network-paper} that it is currently used in the Lightning Network,
and \emph{Decker-Wattenhofer duplex payment channels}\cite{Decker2015fast} where these two proposal try to improve the know state
of the art by allowing more operation between the channels between two parties.

In particular the Poon-Dryja payment channel allow .....


In the other hand, the Decker-Wattenhofer duplex payment channel allow to have a .....

\section{Lightning Network implementation}

The basic idea of the Lightning Network is simple using the underline Bitcoin Script to allow the possibility to exchange
not final transaction between participants. However, the actual Lightning Network implement a tech stack that looks more complicated
as the Figure \ref{fig:lightning-stack} shows:

\begin{figure}[h]
  \begin{center}
  \includegraphics[width=0.6\columnwidth]{imgs/mtln_0601.png}
  \end{center}
  \caption{Diagram provides an overview of these layers and their component protocols.}
  \label{fig:lightning-stack}
\end{figure}

In fact, the Lightning Network implement a new \emph{Peer to Peer} protocol that exchange information between
peer, regarding the actual status of the network. For this reason, the state machine of the Lightning Network
can be very tricky in some specific case.

In the following Section there is describe a general overview of how the Lightning Network is implemented through the
\emph{BOLT} specification\footnote{BOLT: Basis of Lightning Technology (Lightning Network Specifications)}.

\subsection{Lightning Network Specifications}

Since 2017 the lightning network start to be implemented by some company,
and due the hight overview of the Lightning Network state machine definition in \cite{lightning-network-paper}
the implementation start to cooperate in a Specification where all the lightning implementation agree on.

Therefore, the specification describe the following concept to allow full interoperability between implementation:

\begin{itemize}
  \item BOLT 1: describe the basic of the protocol, like how a message is encode and what type of messages are required in order to support the communication;
  \item BOLT 2: describe the peer to peer protocol for channels managements. This is one of the core concept of lightning, in fact after a payment channel
        is settled up, the channels can have different cases that are described in the Section \ref{sec:channel_state}
  \item BOLT 3: describe how the Bitcoin script language is used to compose the
        payment channel, and what kind of Bitcoin transaction are involved.
  \item BOLT 4: describe the onion routing protocol that it is used to exchange message between peers where in this case are called \emph{hop}.
        The routing schema is based on the \emph{Sphinx}\cite{sphinx} construction and is extended with a per-hop payload.
  \item BOLT 5: describe how the on-chain transaction should be handled.
  \item BOLT 7: describe how the peer to peer node and channels are discovery through the network;
  \item BOLT 8: describe all the authenticated and encrypted transport protocol;
  \item BOLT 9: describe how the the feature flags\footnote{Feature flags that are used to establish what kind of feature are supported by a node} need to be managed;
  \item BOLT 10: describe the DNS Bootstrap and Assisted Node Location;
  \item BOLT 11: describe the invoice protocol for the lightnig network protocol;
\end{itemize}

To add a new feature inside the protocol it need to be discussed between implementations and when at least two implementation support the feature, it is considered standard.

The actual implementations that are compliant and contribute to the lightning network specification are:

\begin{itemize}
  \item \emph{core lightning}:
  \item \emph{lnd}:
  \item \emph{ldk}:
  \item \emph{eclair}:
\end{itemize}

\subsection{Lightning Network Channel State}
\label{sec:channel_state}

The channel state management define the state machine of the channels, and they are described in the BOLT 2 specification. Currently, it is a very complex
state machine where there are involved several steps in order to avoid bad
situation where a node can steal funds or lock the funds involve in the blockchain for a long period.

\section{Lightning Network Extension}
