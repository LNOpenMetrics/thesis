\setcounter{page}{1}
\chapter{Lightning Network}

\newcommand{\noteOnBitcoinNaming}[0]{
    \footnote{
        From now till the end of the document we will use the word Bitcoin Protocol and
        Bitcoin with capitalized for identify the protocol, and the word bitcoin not capitalized to identify the currency
    }
}

\newcommand{\noteOnCLNImpl}[0]{
    \footnote{
    core lightning is one of the main implementation today, and it is available at
    {\href{https://github.com/ElementsProject/lightning}{https://github.com/ElementsProject/lightning}}
    }
}



\section{Introduction}

The \emph{Lightning Network} is a peer-to-peer protocol that propose a scalable solution for Bitcoin\noteOnBitcoinNaming with
one of the goal to enable micro transaction with it. In fact, Bitcoin does not scale for a sequence of reason such as: the number of
maximum rate transaction that the network can process.\\
Therefore, with this limitation that are considered feature in Bitcoin protocol, the lightning network try to solve
the problem by offer an \emph{off-chain} solution that enable peers to exchange partial
singned transaction without violate the Bitcoin rules.\\
Lightning Network is proposed for the first in the 2015 in a draft paper \cite{lightning-network-paper},
and the first time implemented in the 2018 with the \emph{c-lightning}\noteOnCLNImpl (aka \emph{core lightning}) implementation.
However, the protocol evolved over the years by a process of standardization\cite{lightning-bolts} and the paper is considered
only a written idea on how to build an off-chain solution, without any definition of the current protocol state machine.

\section{Off-Chain Protocol}
What is an offchain protocol, and why we need it.

\section{Lightning Network Communication Protocol}

How the nodes talks, and how the transactions are propagated.

\section{Lightning Network future evolution}
eltoo