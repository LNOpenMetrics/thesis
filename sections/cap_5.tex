\chapter{Proposed Solution}

\begin{chapquote}{Jim Barksdale}
If we have data, let’s look at data. If all we have are opinions, let’s go with mine.
\end{chapquote}

\section{Introduction}

Perform a data-driven analysis required to develop a collection of tools
that are good enough to run in a lightning network node with daily activity. 
This section is described an open-source framework for the definition and 
the collection of lightning network metrics.

Therefore, in order to implement this framework an analysis of the state 
of the art was required (described in Section \ref{sec:problem_and_state_of_the_art})
to better understand what are the informations that are more difficult 
to get from a researcher's point of view.

From our preliminary study, we found that information on how a node performs 
on a daily basis are difficult to get, due to the following problems:
\begin{itemize}
    \item Requires a direct interaction with \emph{node operators};\footnote{node operators are people that own a lightning node that point to 
    help the network in routing payments.}
    \item Data collection of this kind can lend to some
    that node operators may want not to share.
\end{itemize}

So, in order to work around these problems our proposal defines in a 
public manner what data will be collected, and how these are analyzed. This 
definition process is done through a lnmetrics Request for Comments (RFC) 
called \emph{lnmetrics RFC}. Then, after the data are defined, the collection 
of this data is done with a public server with public API with the possibility 
to self-hosting the server on hardware with at least a Raspberry PI 2 capability.

In addition, in order to incentivize node operators to provide and share the 
information with one of the public servers available we write the tool that 
collect the information on the lightning node with the possibility to run 
in an offline mode, and only if and when the node operators want the data 
are shared with one of the servers chosen (or more than one server).
One of the reasons that make the possibility to run in an offline mode a good
the incentive for node operators to be part of the research is that they 
are seeking a tool to analyze the node performance in order to increase the profit of
their node.

The Figure \ref{fig:lnmetrics_process} shows as the general process of propose 
a solution with our framework looks like.

\begin{figure}
    \begin{center}
      \includegraphics[scale=0.7]{imgs/lnmetrics-workflow-drawio.png}
    \end{center}
    \caption{Example of a process that uses lnmetrics.}
    \label{fig:lnmetrics_process}
\end{figure}


\section{LNMetrics Request for Comments (RFC)}

The LNMetics Request for Comment (RFC) proposed is a more general idea of what 
it is already been done in the Lightning Network protocol definition. In fact,
this RFC is pointing to having the data description of what data are collected in 
order and how these data are used. It is not trying to define the only way to 
define particular metrics on the lightning network, but more to incentivize
discussion between people to achieve a better result.

\subsection{Data Definition}

Data definition is the most important part to propose a solution for a problem 
that required data analysis. Therefore, the data definition is a core 
part of the lnmetrics data definition process as the Figure \ref{fig:lnmetrics_process}
shows. For this reason, the lnmetrics.rfc process is driven by proposing a new 
\emph{metric} that is composed of the following part:

\begin{itemize}
    \item {\bf Metric Introduction}: An Introduction of the area that the metrics are targeting;
    \item {\bf Input Metric}: Data definition of the data that the researchers need. The data definition is done through a \emph{JSON schema}, and
    \item {\bf Output Metric}: Data definition of the data that the research point to return as a result. 
\end{itemize}

Ideally, the Metric proposal needs to be supported by a reference implementation to 
allow people to be part of the research by running the tool provided.
For this reason, the input and output metrics are defined through a JSON schema 
that allows the usage of source code generation tools in order to simplify the development process.
However, at this time the problem on how to easily integrate new metrics inside the 
existing code is an open problem, and a possible solution is to allow the server 
to support a \emph{plugin protocol} that allow any person to extend the existing 
server with additional features such as a new metric with any languages that they preferer.

In conclusion, when the new metric has been proposed inside the RFC a discussion between interested 
people need to be done in order to try to improve and verify the proposal, but if this is not possible
the proposal can be accepted directly by providing a reference implementation of the metrics proposed.

\section{LNMetrics Client Data Collection}

LNMetrics client is a generic concept of a tool that is able to collect one or more metrics 
defined inside the RFC, and allow a node operator or anyone that runs a lightning node
to collect data and share it with an analysis system described in Section \ref{sec:lnmetrics_server}.

In order to support our proposal a generic solution implemented in \emph{Go language} is provided 
and available on Github at \url{https://github.com/LNOpenMetrics/go-lnmetrics.reporter}. In order to 
provide a generic solution and support different kind of metrics, we develop the code with 
Object oriented paradigm and implement the concept of metrics under an interface defined 
in Code \ref{code:lnmetric_client_inter}.

\begin{lstlisting}[language=go, basicstyle=\small,
                  caption={Metric interface provided in our client reference implementation.}, 
                  label={code:lnmetric_client_inter}]
// All the metrics need to respect this interface
type Metric interface {
    // return the unique name that the metrics has
	MetricName() *string
    // called when the metrics is initialized from 
    // the lightning node
    OnInit(lightning client.Client) error
    // called when the node is shutting down
    OnStop(msg *Msg, lightning client.Client) error
    // used to make the actual status of the metrics
    // persistent
    MakePersistent() error
    // called when the metrics is ready to be published
    UploadOnRepo(client *graphql.Client, 
        lightning client.Client) error
    // called when the metrics for the specific node 
    // need to be initialized on the server
    InitOnRepo(client *graphql.Client, 
        lightning client.Client) error
    // call that the lightning node use when it is 
    // time to update the metrics with new data
    Update(lightning client.Client) error
    // a more specific method to be able to call the 
    // update metrics with a specific message that allow 
    // to pass more information to the update method
    UpdateWithMsg(message *Msg, 
        lightning client.Client) error
    ToMap() (map[string]any, error)
    ToJSON() (string, error)
    // allow to perform data migration from an old 
    // data version to a new one
    Migrate(payload map[string]any) error
}
\end{lstlisting}

\section{LNMetrics Data Analsys}
\label{sec:lnmetrics_server}
